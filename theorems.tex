\documentclass{article}
\usepackage{graphicx} % Required for inserting images
\graphicspath{{Images/}}
\usepackage[utf8]{inputenc}
\usepackage{multicol}
\usepackage{amsthm}
\usepackage{amsmath}
\usepackage{xcolor}

\title{Analyse I - Théorèmes à l'examen}
\author{Laura Paraboschi / Simon Lefort }
\date{BA1 - IN}

\begin{document}

\maketitle

\section{Unicité de la limite d'une suite}

Supposons par l'absurde que:
\[ \lim_{n\to\infty}a_n = a\ \text{et} \lim_{n\to\infty}a_n = b,\ \text{avec a}\ \neq b \].\\
Alors:
\[ \lim_{n\to\infty}a_n = a\ \Leftrightarrow \forall \epsilon > 0,\ \exists n_1 \in \mathbf{N^*},\ \text{t.q.}\ \forall n > n_1,\ \lvert a_n - a \lvert\ \leq \frac{\epsilon}{2}\ \textbf{(1)}\]
\[ \lim_{n\to\infty}a_n = b\ \Leftrightarrow \forall \epsilon > 0,\ \exists n_2 \in \mathbf{N^*},\ \text{t.q.}\ \forall n > n_2,\ \lvert a_n - b \lvert\ \leq \frac{\epsilon}{2}\ \textbf{(2)}\]\\
Soit $ n_0 := max(n_1, n_2) $.\\
$ \forall n \geq n_0 $, on a à la fois \textbf{(1)} et \textbf{(2)}.\\
Puisque $ a - b = (a - a_n) + (a_n - b) $, alors $ \forall n \geq n_0 $:\\
\[ 0\ \leq \lvert a - b \lvert\ \leq \lvert a - a_n \lvert\ +\ \lvert a_n - b \lvert\ \leq \frac{\epsilon}{2} + \frac{\epsilon}{2} = \epsilon \]
On a donc que $ \forall \epsilon > 0,\ 0 \leq \lvert a - b \lvert\ \leq \epsilon $ par l'axiome d'Archimède.\\\\
En effet, comme $ |a-b| \in \mathbf{R} $, et $ \mathbf{R} $ est archimédien, alors l'axiome d'Archimède affirme que si $ p \in R\ \text{et}\ \forall n \in \mathbf{N*},\ 0 \leq p \leq \frac{1}{n} \implies p = 0 $. \\\\
On pose ici $ p = |a - b| \text{ et } \frac{1}{n} = \epsilon $, donc $ a - b = 0 $, donc $ a = b $ ce qui contredit notre supposition de départ.

\newpage

\section{Théorème des gendarmes}

\textbf{Hypothèses} \\
\[ \lim_{n\to\infty}a_n = \lim_{n\to\infty}b_n = c \]
\[ a_n < C_n < b_n\ \forall n \geq m \]
\textbf{Alors} \\
$\lim_{n\to\infty}c_n = c$
\[ \forall \epsilon > 0,\ \exists n_0 \geq m\ \text{t.q.}\ \forall n \geq n_0\ \text{:} \]
Démonstration :\\\\
On a $a_n \leq c_n \leq b_n \Leftrightarrow a_n - c \leq c_n - c \leq b_n - c \quad \forall n \geq m$ par l'hypothèse 2 \\
\begin{itemize}
    \item $\forall \epsilon > 0,\exists n_1 \in \mathbf{N}$ tel que $\forall n\geq n_1 \lvert a_n - c \lvert < \epsilon$ puisqu'an converge
    \item $\forall \epsilon > 0,\exists n_2 \in \mathbf{N}$ tel que $\forall n\geq n_2 \lvert b_n - c \lvert < \epsilon$ puisque bn converge
\end{itemize}
Posons N = max(m, $n_1, n_2)$, alors : \\\\
$a_n - c \leq c_n - c \leq b_n - c \quad \forall n \geq N \geq m $\\\\
$\forall \epsilon > 0,\exists N \geq m$ tel que $\forall n\geq N$
\begin{itemize}
    \item $ \lvert a_n - c \lvert < \epsilon \implies -\epsilon \leq a_n - c \leq \epsilon$
    \item $ \lvert b_n - c \lvert < \epsilon \implies -\epsilon \leq b_n - c \leq \epsilon$
\end{itemize}
Par la définition de la limitie et de N = max(m, $n_1, n_2)$ \\\\
$ \implies \forall n\geq N$ \\
$-\epsilon \leq a_n - c \leq c_n - c \leq b_n - c \leq \epsilon \implies -\epsilon\leq c_n - c \leq \epsilon $ \\\\
$\implies \forall n \geq N, \lvert c_n - c \lvert < \epsilon \Leftrightarrow \lim_{n\to\infty}c_n = c$

\newpage

\section{Si une série converge, la suite des termes tend vers 0}

On veut démontrer :
\[ \sum_{k=0}^{\infty} a_k\ \text{converge} \implies \lim_{n\to\infty} a_n = 0 \]\\
Si la série est convergente, alors la suite des sommes partielles $ S_n $ est aussi convergente, et est donc une suite de Cauchy.
\[ S_n = \sum_{k=0}^{n} a_k \]\\
alors :
\[ \forall \epsilon,\ \exists\ N\ \text{t.q}\ \forall n, m \geq N \in \mathbf{N},\ \lvert S_n - S_m \lvert\ \leq \epsilon \]
en particulier, on a :
\[ \lvert S_{m+1} - S_m \lvert\ \leq \epsilon \]
\[ \Leftrightarrow \lvert a_{m+1} \lvert\ \leq \epsilon \]\\\\
Ainsi, on retrouve la définition de limite pour $ a_n $:\\
\[ \forall\ \epsilon,\ \exists\ n_0 \in \mathbf{N},\ \text{t.q}\ \forall n \geq n_0,\ |a_n - 0| \leq \epsilon \]\\
\[ \Leftrightarrow \lim_{n\to\infty} a_n = 0 \]

\newpage

\section{Prouver que la limite d'une fonction f donnée est 0}

On nous donne la suite $ f(x) = 0 $ si $ x \neq 0 $, et $ f(0) = 1 $.\\
On veut démontrer que $ \lim_{x\to{0}} f(x) = 0 $. (et non pas 1...)\\\\
\textbf{Définiton de la limite épointée :}\\
$ f : D \to \mathbf{R} $ admet pour limite épointée $ l \in \mathbf{R} $ lorsque $ x \to x^* $ si $ \forall (x_n) $ t.q $ \forall n, x_n \in D \backslash \{x^*\} $ t.q $ \lim_{n\to\infty} x_n = x^*$, la suite $ y_n = f(x_n) $ converge et $ \lim_{n\to\infty} y_n = l$\\\\
Ici notre $ x^* = 0 $, donc on choisit une suite $ x_n $ telle que $ lim_{n\to\infty} x_n = 0$, et $ x_n \neq 0$. On a donc $ \forall n, y_n = f(x_n) = 0 $.\\\\
Ainsi, on peut connaître $ \lim_{n\to\infty} y_n = l = \lim_{n\to\infty} 0 = 0$.

\newpage
\section{Prouver que la limite de $ sin(\frac{1}{x}) $ en 0 n'existe pas}

On nous donne la suite $ f(x) = sin(\frac{1}{x}) $.\\
On veut démontrer que $ \lim_{x\to{0}} f(x) $ n'existe pas\\\\
\textbf{Définiton de la limite épointée :}\\
$ f : D \to \mathbf{R} $ admet pour limite épointée $ l \in \mathbf{R} $ lorsque $ x \to x^* $ si $ \forall (x_n) $ t.q $ \forall n, x_n \in D \backslash \{x^*\} $ t.q $ \lim_{n\to\infty} x_n = x^*$, la suite $ y_n = f(x_n) $ converge et $ \lim_{n\to\infty} y_n = l$\\\\
Ici notre $ x^* = 0 $. On doit trouver une suite $ x_n $ t.q $ \lim_{n\to\infty} = 0 $ et $ x_n \neq 0$, mais t.q $ y_n = f(x_n) $ diverge.\\\\
On pose $ x_n = \frac{1}{\frac{\pi}{2} + n \cdot \pi}$. On a $ y_n = sin(\frac{\pi}{2} + n \cdot \pi) = (-1)^n $ qui ne converge pas.

\end{document}
