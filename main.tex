\documentclass{article}
\usepackage{graphicx} % Required for inserting images
\graphicspath{{Images/}}
\usepackage[utf8]{inputenc}
\usepackage{multicol}
\usepackage{amsthm}
\usepackage{amsmath}
\usepackage{xcolor}

\newtheorem{definition}{Definition}[section]
\newtheorem{remark}{Remarque}[section]
\newtheorem{theorem}{Théorème}

\title{Analyse I}
\author{Laura Paraboschi / Simon Lefort }
\date{BA1 - IN}

\begin{document}

\maketitle

\section{Règles de calcul}

\begin{multicols}{2}
\subsection{Puissance}  
\columnbreak
\subsection{Logarithme}
\end{multicols}

\begin{multicols}{2}

\begin{itemize}
    \item \( a^n \cdot a^m = a^{m+n} \)
    \item \( (ab)^m = a^mb^m \)
    \item \( (\frac{a}{b})^m = (\frac{a^m}{b^m}) \)
    \item \( \frac{a^m}{a^n} = a^{m-n} \)
\end{itemize}
\columnbreak

\begin{itemize}
    \item \( \log_a(x \cdot y) = \log_a(x) + \log_a(y) \)
    \item \( \log_a(\frac{1}{x}) = -\log_a(x) \)
    \item \( \log_a(\frac{x}{y}) = \log_a(x) - \log_a(y) \)
    \item \( \log_a(x^r) = r\log_a(x) \)
\end{itemize}

\end{multicols}

\section{Ensemble}
\begin{definition}
P(X) est l'ensemble dont les éléments sont tous les sous-ensembles de X. Sa capacité est de  \(2^n\), avec n = nombres d'élements dans X
\end{definition}
\subsection{Produit cartésien}
\[ X \times Y \times Z := \{\,(x,y,z)\quad |\quad x \in X,\,y \in Y,\, z \in Z\, \} \]
\begin{definition}
    Un sous ensemble \(R \subset X \times Y\) est appelé \underline{une relation binaire} sur X et Y
\end{definition}
\begin{definition}
    Un sous ensemble \(R \subset X \times X\) est appelé \underline{une relation} sur X
\end{definition}

\subsection{Classe d'équivalence}
\begin{definition}
    Un sous ensemble \(R \subset X \times X\) est appelé \underline{une relation d'équivalence} sur X si :
    \begin{enumerate}
        \item \( \forall x \in X,\quad x \sim x \) (R est réfléctive)
        \item \( \forall x,\,y \in X,\quad  x \sim y \Rightarrow y \sim x \) (R est symétrique)
        \item \( \forall x,\,y,\, z \in X,\quad x \sim y\, \wedge\, y \sim z \Rightarrow x \sim z \) (R est transitive)
    \end{enumerate}
\end{definition}
\begin{definition}
    On définit \(C_x \subset X \) par \\ \[ C_x := \{\, y \in X\quad |\quad y \sim x\, \} \] \\
    \(C_x\) est appelé la classe d'équivalence de x
\end{definition}
\begin{definition}
    L'ensemble quotient \(X/_\sim \) est l'ensemble des classes d'équivalences distinctes de X
\end{definition}
\subsection{Fonction}
\begin{definition}
    Une fonction f : \(D \rightarrow Y\) est appelée 
    \begin{itemize}
        \item \underline{surjective} si Im(f) = Y
        \item \underline{injective} si \( f(x_1) = f(x_2) \Rightarrow x_1 = x_2 \)
    \end{itemize}
\end{definition}
\begin{remark}
    \( (A \Rightarrow B) \Leftrightarrow (\overline{A} \Rightarrow \overline{B}) \) \quad Une proposition contraposée
\end{remark}
\begin{remark}
    Toute fonction f : \(D \rightarrow Y\) définit une fonction surjective\\ \[f : D \rightarrow Im(f) \subset Y\]
\end{remark}
\begin{definition}
    Une fonction qui est surjective et injective est appelée bijective
\end{definition}
\begin{remark}
    Toute fonction bijective \(D \rightarrow Y\) possède une fonction réciproque notée \( f^{-1}, Y \rightarrow D\)
\end{remark}
Soient deux fonctions f :\(D \rightarrow Y\) et g : \(E \rightarrow Y\) avec \(E \subset D\), telle que pour tout \(x \in E, g(x) = f(x)\). Alors :
\begin{itemize}
    \item g est appelée la restriction de f à E : \(g = f|_E\)
    \item f est appelée un prolongement de g de E à D
\end{itemize}
\begin{definition}
    Le graphe d'une fonction f : \(D \rightarrow Y\) est l'ensemble
    \[ G_f := \{\, (x,y) \in D \times Y\quad |\quad y = f(x)\, \} \]
    \qquad Attention, pour tout x, il ne doit exister \textcolor{red}{qu'une seule} image (qu'un seul y = f(x)).
\end{definition}
\begin{definition}
    On définit \(h = g \circ f\) comme étant la composition de fonction 
    \[h = g(f(x))\]
    \qquad La loi de la composition de fonction est associative
\end{definition}
\newpage
\section{Plus grand commun diviseur}
\subsection{Algorithme de J. Stein}
\begin{itemize}
    \item pgcd(a,b) = pgcd(b,a)
    \item pgcd(a,b) = 2\(\cdot\) pgcd(\(\frac{a}{2},\frac{b}{2})\) \qquad si a, b pairs
    \item pgcd(a,b) = pgcd(\(\frac{a}{2}\),b) \qquad si a pair, b impair
    \item  pgcd(a,b) = pgcd(\(\frac{a-b}{2}\), b)  \qquad si a, b impairs et a \(\geq \) b
    \item pgcd(a,0) = a
\end{itemize}
\section{Raisonnement par récurrence}
\begin{theorem} 
\end{theorem}
\begin{enumerate} 
    \item Si P(\(n_0)\) est vrai pour un \(n_0 \in \mathbf{N}\)
    \item Si pour tout n \(\geq n_0\), P(n) \(\Rightarrow\) P(n+1)
\end{enumerate}
\qquad \textit{Alors P(n) est vrai pour tout n \(\geq n_0\)}
\section{Les notations \(\Sigma\) et \(\Pi\)}
\(\sum_{k=m}^n a_k := a_m + a_{m+1} + ... + a_{n-1} + a_n \) \\\\
\( \prod_{k=m}^{m} a_k := a_m \cdot a_{m+1} \cdot ... \cdot a_{n-1} \cdot a_n\)\\\\
\( Si\ n < m, alors \qquad \sum_{k=m}^{n} a_k := 0,\qquad \prod_{k=m}^{n} a_k = 1 \)
\subsection{Règles de calcul}
\( \sum_{k=l}^{m} a_k + \sum_{k=m+1}^{n} a_k = \sum_{k=l}^{n} a_k \)\\\\
\( (\prod_{k=l}^{m} a_k)\cdot(\prod_{k=m+1}^{n} a_k) =\prod_{k=l}^{n} a_k \)\\\\
\( \sum_{k=m}^{n}(a_k + b_k) = \sum_{k=m}^{n} a_k + \sum_{k=m}^{n} b_k \)\\\\
\( \prod_{k=m}^{n}(a_k \cdot b_k) = (\prod_{k=m}^{n} a_k) \cdot (\prod_{k=m}^{n} b_k)\)
\subsection{Identité}
\(\sum_{k=0}^{n} a^k = \frac{1-a^{n+1}}{1-a}, \qquad a \neq 1\)\\\\
\( \sum_{k=0}^{n} k = \frac{n\cdot(n+1)}{2} \)\\\\
\( (a + b)^2 = a^2 + 2ab + b^2  \)\\\\
\( (a + b)^3 = a^3 + 3a^2b + 3ab^2 + b^3 \)\\\\
\( (a + b)^n = \sum_{k=0}^{n} 
\begin{pmatrix}
    n \\ k
\end{pmatrix} a^{n-k}\cdot b^k = a^n + na^{n-1}b + ... + b^n \)\\\\
On peut tout déduire du triangle de Pascal :
\begin{figure}[htp]
    \centering
    \includegraphics[width=8cm]{Images/Pascal-Triangle.png}
    \caption{Triangle de Pascal}
\end{figure}
\end{document}

